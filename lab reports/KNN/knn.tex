\documentclass[12pt]{article}
\usepackage[UTF8]{ctex}
\usepackage[margin=1in]{geometry}
\usepackage{graphicx} % For including figures
\usepackage{amsmath}  % For math fonts, symbols and environments
\usepackage{pgfgantt} % For Gantt charts
\usepackage[hidelinks]{hyperref} % For hyperlinks
\usepackage{enumitem} % For customizing lists
\definecolor{blue}{HTML}{74BBC9}
\definecolor{yellow}{HTML}{F7E967}
\usepackage{listings}
\usepackage{xcolor}
\usepackage{tocloft} % 导入tocloft包
\usepackage{zi4}
\usepackage{fontspec}

\usepackage{graphicx} % For including graphics
\usepackage{booktabs} % For professional looking tables
\usepackage{array}    % For extended column definitions
\usepackage{amsmath}  % For math environments like 'equation'
\usepackage{amsfonts} % For math fonts like '\mathbb{}'
\usepackage{amssymb}  % For math symbols
\usepackage{caption}  % For custom captions
\usepackage[table]{xcolor} % For coloring tables
% 目录标题样式定义
\renewcommand{\cfttoctitlefont}{\hfill\Large\bfseries}
\renewcommand{\cftaftertoctitle}{\hfill\mbox{}\par}

% Monokai theme with a lighter background
\definecolor{codegreen}{rgb}{0,0.6,0}
\definecolor{codegray}{rgb}{0.5,0.5,0.5}
\definecolor{codepurple}{rgb}{0.58,0,0.82}
\definecolor{backcolour}{rgb}{0.95,0.95,0.92}
\setmonofont{Source Code Pro}[Contextuals={Alternate}]

\lstdefinestyle{mystyle}{
    backgroundcolor=\color{backcolour},   
    commentstyle=\color{codegreen},
    keywordstyle=\color{magenta},
    numberstyle=\tiny\color{codegray},
    stringstyle=\color{codepurple},
    basicstyle=\ttfamily\footnotesize,
    breakatwhitespace=false,         
    breaklines=true,                 
    captionpos=b,                    
    keepspaces=true,                 
    numbers=left,                    
    numbersep=5pt,                  
    showspaces=false,                
    showstringspaces=false,
    showtabs=false,                  
    tabsize=2
}

\lstset{style=mystyle}




\title{\textbf{PR Experiment 1 Experiment Report}}
\author{58122204 谢兴}
\date{\today}

\begin{document}


\begin{titlepage}

  \begin{figure}[htbp]
    \centering
    \includegraphics[scale=0.2]{figures/southeast_university_logo.png}
    % \caption{实际投掷距离与理想条件下投掷距离对比}
    \label{fig1}
  \end{figure}


  \centering
  \vspace*{40pt}
  \Huge\textbf{模式识别实验报告}

  \vspace{60pt}
  \Large
  实验题目:KNN

  \vspace{25pt}
  学院:计算机科学院工程学院、软件学院、人工智能学院

  \vspace{25pt}
  专业:人工智能

  \vspace{25pt}
  姓名:谢兴

  \vspace{25pt}
  学号:58122204

  \vspace{25pt}
  完成日期:2024年5月26日

  \vspace{25pt}
  指导老师:薛晖





\end{titlepage}


\newpage
\tableofcontents


\section*{实验目的}
This assignment is intended to learn how to create, work with and manipulate processes in Linux.
You are expected to refer to the text book and references mentioned in the course website before you start the lab.







\newpage
% 开始附录部分
\appendix
% 手动添加附录到目录中
\addcontentsline{toc}{section}{附录A}
\section*{实验内容一Code}
% 附录内容...
% ... 附录内容 ...
\subsection*{cpu\_bound.c}
\begin{lstlisting}[language=C]
#include <stdio.h>

// Function to calculate factorial
unsigned long long factorial(int n) {
    if (n == 0) return 1;
    return n * factorial(n - 1);
}

int main() {
    int num = 10;  // A small number to keep calculations quick but repetitive

    while(1) {
        factorial(num);
    }

    return 0;
}

\end{lstlisting}



\end{document}
