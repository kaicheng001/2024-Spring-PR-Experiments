\documentclass[12pt]{article}
\usepackage[UTF8]{ctex}
\usepackage{xeCJK}
\usepackage[margin=1in]{geometry}
\usepackage{graphicx} % For including figures
\usepackage{amsmath}  % For math fonts, symbols and environments
\usepackage{pgfgantt} % For Gantt charts
\usepackage[hidelinks]{hyperref} % For hyperlinks
\usepackage{enumitem} % For customizing lists
\definecolor{blue}{HTML}{74BBC9}
\definecolor{yellow}{HTML}{F7E967}
\usepackage{listings}
\usepackage{xcolor}
\usepackage{tocloft} % 导入tocloft包
\usepackage{zi4}
\usepackage{fontspec}
\usepackage{setspace} % For setting line spacing

\usepackage{booktabs} % For professional looking tables
\usepackage{array}    % For extended column definitions
\usepackage{amsfonts} % For math fonts like '\mathbb{}'
\usepackage{amssymb}  % For math symbols
\usepackage{caption}  % For custom captions
\usepackage[table]{xcolor} % For coloring tables

% 目录标题样式定义
\renewcommand{\cfttoctitlefont}{\hfill\Large\bfseries}
\renewcommand{\cftaftertoctitle}{\hfill\mbox{}\par}

% 设置中文主字体为宋体,指定相对路径
\setCJKmainfont[
    Path = ./,
    BoldFont = SimSun.ttc,
    ItalicFont = SimSun.ttc
]{SimSun.ttc}

% 设置英文主字体为Times New Roman
\setmainfont{Times New Roman}

% 设置正文格式:宋体,小四,行距20磅
\renewcommand\normalsize{%
    \CJKfamily{song}\fontsize{12pt}{20pt}\selectfont}

% Monokai theme with a lighter background
\definecolor{codegreen}{rgb}{0,0.6,0}
\definecolor{codegray}{rgb}{0.5,0.5,0.5}
\definecolor{codepurple}{rgb}{0.58,0,0.82}
\definecolor{backcolour}{rgb}{0.95,0.95,0.92}
\setmonofont{Source Code Pro}[Contextuals={Alternate}]

\lstdefinestyle{mystyle}{
    backgroundcolor=\color{backcolour},   
    commentstyle=\color{codegreen},
    keywordstyle=\color{magenta},
    numberstyle=\tiny\color{codegray},
    stringstyle=\color{codepurple},
    basicstyle=\ttfamily\footnotesize,
    breakatwhitespace=false,         
    breaklines=true,                 
    captionpos=b,                    
    keepspaces=true,                 
    numbers=left,                    
    numbersep=5pt,                  
    showspaces=false,                
    showstringspaces=false,
    showtabs=false,                  
    tabsize=2
}

\lstset{style=mystyle}

\title{\textbf{PR Experiment 1 Experiment Report}}
\author{58122204 谢兴}
\date{\today}

\begin{document}

\begin{titlepage}
  % \begin{figure}[htbp]
  %   \centering
  %   \includegraphics[scale=0.2]{figures/southeast_university_logo.png}
  %   % \caption{实际投掷距离与理想条件下投掷距离对比}
  %   \label{fig1}
  % \end{figure}

  \centering
  \vspace*{40pt}
  \Huge\textbf{模式识别实验报告}

  \vspace{60pt}
  \Large
  实验题目:KNN

  \vspace{25pt}
  学院:计算机科学院工程学院、软件学院、人工智能学院

  \vspace{25pt}
  专业:人工智能

  \vspace{25pt}
  姓名:谢兴

  \vspace{25pt}
  学号:58122204

  \vspace{25pt}
  完成日期:2024年5月26日

  \vspace{25pt}
  指导老师:薛晖
\end{titlepage}

\newpage
\tableofcontents

\section*{实验目的}
利用KNN算法,对 Iris 鸢尾花数据集中的测试集进行分类。

\section{任务说明}
\begin{enumerate}
  \item 利用欧式距离作为KNN算法的度量函数,对测试集进行分类。实验报告中,要求在验证集上分析近邻数$k$对KNN算法分类精度的影响。
  \item 利用马氏距离作为KNN算法的度量函数,对测试集进行分类。马氏距离是一种可学习的度量函数,详情请参考课题说明文档。注意:实验报告中请对优化过程的梯度计算公式进行推导。
  \item 基于MindSpore平台提供的官方模型库,对相同的数据集进行训练,并与自己独立实现的算法对比结果(包括但不限于准确率、算法迭代收敛次数等指标),并分析结果中出现差异的可能原因,给出使用MindSpore的心得和建议。
  \item (加分项)使用MindSpore平台提供的相似任务数据集(例如,其他的分类任务数据集)测试自己独立实现的算法并与MindSpore平台上的官方实现算法进行对比,并进一步分析差异及其成因。
\end{enumerate}

\section{问题描述}

\section{实验步骤与流程}

\section{实验结果与分析}

\section{实验心得体会}

\newpage
% 开始附录部分
\appendix
% 手动添加附录到目录中
\addcontentsline{toc}{section}{附录A}
\section*{实验内容一Code}

\begin{lstlisting}[language=Python]

\end{lstlisting}

\end{document}
